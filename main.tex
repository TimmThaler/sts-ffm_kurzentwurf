\documentclass{kurzentwurf}

%%%%%%%%%%%%%%%%%%%%%%%%%%%%%%%%%%%%%%%%%%%%%%%%%%%%%%%%%%%%%%%%%%%%%%
%------------------------------ Setup -------------------------------%
%%%%%%%%%%%%%%%%%%%%%%%%%%%%%%%%%%%%%%%%%%%%%%%%%%%%%%%%%%%%%%%%%%%%%%

\name{Musterfrau}
\vorname{Martina}
\schule{Tolle Schule}
\datum{01.01.70}
\stunde{42}
\zeit{08:00 Uhr}
\raum{1337}
\modul{Fachmodul (Ausbilder*in)}
\gaeste{Weitere Personen}
\lerngruppe{Klasse 10c}
\anzahl{33}

%%%%%%%%%%%%%%%%%%%%%%%%%%%%%%%%%%%%%%%%%%%%%%%%%%%%%%%%%%%%%%%%%%%%%%

\begin{document}

\maketitle

\section{Fragestellung / Problemstellung / Thema der Unterrichtsstunde // Didaktisches Zentrum} % max. 3 – 4 Sätze

\section{Zusammensetzung der Lerngruppe: Gruppendynamische Besonderheiten, Lernatmosphäre etc.}
\section{Diagnose des Lernstands: Vorkenntnisse und Kompetenzen bezogen auf das Unterrichtsvorhaben}

2.+3.: Zusammen max. 1,5 Seiten.

\section{Didaktische Begründung der Zielsetzung der Stunde} % max. 1 Seite


\section{Tabellarische Übersicht} % ca. 1 Seite

\begin{uebersicht}
    \phase{Begrüßung (1')}{Die LiV begrüßt die Schüler.}{Die Schüler begrüßen die LiV und die Gäste.}{LSG}
    \phase{Einstieg (3')}{Die LiV schildert die Planung der Stunde.}{}{Plenum} 
\end{uebersicht}

\section{Anlagen}
Sitzplan aus der Perspektive der Gäste, Literaturhinweise, Materialien


%%%%%%%%%%%%%%%%%%%%%%%%%%%%%%%%%%%%%%%%%%%%%%%%%%%%%%%%%%%%%%
\pagebreak
\printbibliography[title={Literatur}]

\end{document}
