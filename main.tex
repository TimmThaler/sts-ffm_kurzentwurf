\documentclass{kurzentwurf}

\name{Musterfrau}
\vorname{Martina}
\schule{Tolle Schule}
\stunde{42}
\zeit{01.01.70, 14:00 Uhr}
\raum{1337}
\modul{Fachmodul (Ausbilder*in)}
\gaeste{Weitere Personen}
\lerngruppe{Klasse 10c}
\anzahl{33}

\begin{document}

\maketitle

\section{Thema der Unterrichtsstunde}


\section*{Didaktisches Zentrum}
\section{Zusammensetzung der Lerngruppe: Gruppendynamische Besonderheiten, Lernatmosphäre etc.}
\section{Diagnose des Lernstands: Vorkenntnisse und Kompetenzen bezogen auf das Unterrichtsvorhaben}
\section{Didaktische Begründung der Zielsetzung der Stunde}

\section{Tabellarische Übersicht}

\section{Anlagen: Sitzplan aus der Perspektive der Gäste}
\section*{Literaturhinweise, Materialien}


%%%%%%%%%%%%%%%%%%%%%%%%%%%%%%%%%%%%%%%%%%%%%%%%%%%%%%%%%%%%%%
\pagebreak
\printbibliography[title={Literatur}]

\end{document}
